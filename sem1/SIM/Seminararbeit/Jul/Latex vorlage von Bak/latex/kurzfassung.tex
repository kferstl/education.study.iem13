\chapter{Kurzfassung}
Um Roboter im Innenraum betreiben zu können, wird ein Positionierungssystem benötigt. Globale Systeme wie GPS funktionieren im Innenraum nur sehr eingeschränkt. Im Zuge dieser Arbeit wird daher ein lokales Positionierungssystem auf der Basis von Ultraschall geplant. Bei dem vorgeschlagenen System können Ultraschall-Empfänger ihre Position, mit entsprechender Signalverarbeitung, durch im Raum verteilte Ultraschall-Sender bestimmen. Dies geschieht mittels Laufzeitmessung, dazu wird das TDOF \emph{time difference of arrival} Verfahren verwendet. Es wird gezeigt, dass sich Ultraschall bestens zur Positionsbestimmung auf kleinem Raum eignet und eine kostengünstige Lösung möglich ist. Um möglichst genaue Positionen bestimmen zu können, wird CDMA \emph{Code Division Multiple Access} verwendet. Dadurch können alle Sender gleichzeitig senden und der Empfänger kann alle Signale richtig zuordnen. Im Zuge der Arbeit wurden Teile des Senders, sowie des Empfängers implementiert, um die Machbarkeit nachzuweisen. Die gesamte Signalverarbeitung geschieht digital in einem FPGA. Unterschiedliche Verfahren sowie Alternativen wurden zusätzlich beschrieben und verglichen.

\chapter{Abstract}

\begin{english} %switch to English language rules

Robots operating indoors often need to calculate their position inside a room. Therefore a positioning system is required. Global systems like GPS don’t really work indoors. The aim of this work is to plan a local position system based on ultrasound waves. This system enables ultrasound receivers to calculate the position using some transmitters spread across the room, by using special signal analysis. This is done by time of flight measurements a special algorithm called TDOF \emph{time difference of arrival} is used. This work shows that ultrasound is ideal to calculate positions inside a small room and it is also quite cost efficient. To calculate positions as accurate as possible CDMA \emph{Code Division Multiple Access} is used. Using CDMA it is possible that all transmitters transmit at the exact same time and the receiver is still possible to identify each signal. There is also a proof of concept implementation for the receiver and transmitter. The whole signal processing is completely done digitally using an FPGA. Furthermore different techniques and alternatives are shown and discussed.

\end{english}
