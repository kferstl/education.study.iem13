\chapter{Inhalt der CD-ROM}
\label{app:cdrom}

\paragraph{Format:} 
		CD-ROM, Single Layer, ISO9660-Format (Standart)%
%\footnote{Verwenden Sie möglichst ein Standardformat, bei DVDs natürlich
%eine entsprechende andere Spezifikation.}

Im Anhang der Bachelorarbeit befindet sich eine CD-ROM welche die Arbeit im PDF-Format sowie im \latex-Format beinhaltet.

Des weiteren befinden sich auf der CD-ROM der erstellte VHDL Code mit allen Aufzeichnungen.

\section{Bachelorarbeit}
Die Bachelorarbeit in digitaler Form.
\begin{FileList}{/Bachelorarbeit}
\fitem{_DaBa.pdf} Diese Diplomarbeit im PDF-Format.
\fitem{latex/} Dieser Ordner beinhaltet die \latex version der Arbeit.%
\end{FileList}

\section{VHDL-Code}
Beinhaltet den VHDL Code für den Sender sowie Empfänger.
\begin{FileList}{/srcVHDL}
\fitem{XXX.vhd} Dies und Das %
\end{FileList}

\section{C-Code}
Erster versuch den Sender in C zu schreiben.
Dabei wurde schnell klar das VHDL der richtige Ansatz ist.
\begin{FileList}{/srcC}
\fitem{XXX.c} Dies und Das %
\end{FileList}

\section{MATLAB}
Dieser Order beinhaltet alle MATLAB Dateien welcher zur Berechnung der Filter sowie zum erstellen der GOLD-Sequenzen erstellt wurden.
\begin{FileList}{/matlab}
\fitem{XXX.m} Dies und Das %
\end{FileList}

\section{HARDWARE}
Schaltpläne für Empfänger / Sender.
\begin{FileList}{/hardware}
\fitem{XXX.sch} Dies und Das %
\end{FileList}

\section{Dokumentation / Referenzen}
Datenblätter der verwendeten Bauteile.
\begin{FileList}{/bauteile}
\fitem{XXX.sch} Dies und Das %
\end{FileList}

Als Referenz verwendete Publikationen.
\begin{FileList}{/papers}
\fitem{XXX.sch} Dies und Das %
\end{FileList}

Aufgezeichnete Daten Oszilloskope, SignalTrapp, Matlab ...
\begin{FileList}{/aufzeichnungen}
\fitem{XXX.sch} Dies und Das %
\end{FileList}

\section{Praktikumsbericht}
\begin{FileList}{/Praktikumsbericht}
\fitem{_PrBericht.pdf} Der Praktikumsbericht im PDF-Format.
\fitem{latex/} Dieser Ordner beinhaltet die \latex version des Berichtes.%
\end{FileList}