\chapter{Technische Informationen}
\label{ch:TechnischeInfos}

\newcommand*{\checkbox}{{\fboxsep 1pt%
\framebox[1.30\height]{\vphantom{M}\checkmark}}}


\section{Voraussetzungen zum erstellen des VHDL Codes}

Um den VHDL Code synthetisieren zu können wird ein Compiler benötigt.
Wird ein FPGA verwendet so bietet es sich an den Compiler des jeweiligen Herstellers zu verwenden.
Im Fall dieser Arbeit wurde ein Altera FPGA verwendet und dadurch die Entwicklungsumgebung Quartus II von Altera verwendet.
\\\\Altera bietet eine gratis Version von Quartus II an die so genannte Web Version.\footnote{\url{http://www.altera.com/products/software/quartus-ii/web-edition/qts-we-index.html}}
\\\\Um den LogicAnalyser SignalTapII in der Web Version verwenden zu können muss die Talkback Funktion in den Einstellungen aktiviert werden.
\\\\Als Ergänzung zur jeweiligen Entwicklungsumgebung bietet sich Sigasi an ein sehr guter und stabiler VHDL-Editor mit unter Anderem automatischer Vervollständigung und Echtzeit Fehleranzeige.
\\\\Es ist ein kostenlose Universitätsversion von Sigasi verfügbar.\footnote{\url{http://www.sigasi.com/university}}
\\\\Um den VHDL Code zu simulieren wird die Software Questasim oder Modelsim benötigt.
Auch für diese Software ist eine kostenlose Studentenversion verfügbar, nur für Modelsim.\footnote{\url{http://www.mentor.com/company/higher_ed/modelsim-student-edition}}

%\section{Voraussetzungen zum erstellen des Arm-C Codes}
%Dazu werden grundsätzlich 2 Tools benötigt. Einerseits eine Toolchain diese beinhaltet Compiler Linker Debugger usw. sowie eine IDE sprich eine Entwicklungsumgebung.
%\\\\Es gibt eine sehr große Anzahl verschiedener Toolchains am einfachsten zu verwenden sind die YAGARTO Tools. Diese können kostenlos verwendet werden.\footnote{\url{http://www.yagarto.org/}}
%Eine einfache Installation ist ausreichend, es muss nichts weiter beachtet werden.
%\\\\Bezüglich der Entwicklungsumgebung ist es sehr ähnlich es gibt eine Vielzahl meist sehr teurer Tools.
%Eine sehr gute und auch noch kostenlose IDE ist die CoIDE von CooCox.\footnote{\url{http://www.coocox.org/CooCox_CoIDE.htm}}.
%Diese muss lediglich heruntergeladen und installiert werden.

\section{Weitere benötigte Tools}
\begin{itemize}
		\item Matlab\footnote{\url{http://www.mathworks.de/products/matlab/}} mit Simulink sowie dem fdatool zum berechnen der Filter und der Goldsequenzen,
		\item Für die \latex-Version eine funktionierende UTF-8 \latex-Umgebung.
\end{itemize}


