\chapter{Kurzfassung}

Deep Learning-Algorithmen biete eine Möglichkeit vielschichtige Neuronale Netze zu trainieren. Solche Netze kommen in aktuellen Anwendungen zu Erkennung von Objekten in Bildern, Text und Inhalt in Sprache und vielen anderen Gebieten zum Einsatz. Forschungen an den Algorithmen ermöglichen es, die sehr rechenintensiven Algorithmen zu vereinfachen und in großem Stil auf parallelisierten Computersystemen auszuführen. So ist es in einigen Gebieten bereits möglich mit neuronalen Netzen bessere Ergebnisse als mit optimierten Mathematischen verfahren zu erzielen. In folgendem werden die wesentlichen Algorithmen und Strukturen von neuronalen Netzen aufgearbeitet und die Schwierigkeiten und Lösungsansätze aufgezeigt. Außerdem werden einige aktuelle Anwendungen aus der Bild- und Sprachanalyse betrachtet.