\chapter{Einleitung}
\label{cha:einleitung}
% 2 seiten

\section{Motivation}

Komplexe neuronale Netze können bei einigen Problemstellungen, besonders in der Bild- und Spracherkennung ein sehr mächtiges Mittel zur Problemlösung sein. Das Lernen solcher Netze bringt Schwierigkeiten mit sich die in den vergangenen Jahrzenten schlecht oder gar nicht gelöst werden konnten.

Aufgrund der Weiterentwicklung von PC-Hardware und der Verwendung der Grafik-Recheneinheit (GPU), sind seit den 00er-Jahren bereits bemerkenswerte Ergebnisse möglich. Momentan arbeiten sogar Chiphersteller an preiswerten, dedizierten Chips mit trainierbaren neuronalen Netzen. Qualcomm wird den ersten kommerziellen Chip mit einem integrierten neuronalen Netz noch dieses Jahr (2014) veröffentlichen. Neuronale Netze sind immer nur so gut, wie sie trainiert werden, eines der wesentlichsten Themen im weiteren Fortschritt von neuronalen Netzen sind daher die Algorithmen zum Trainieren.

Aus der eingeleiteten Motivation ergeben sich daher folgende Kernaufgaben für die Seminararbeit:

\begin{itemize}
\item Einführung in die Problematik von Deep Learning mit kurzem Blick auf die bisherige Geschichte
\item Analyse der Hürden im Deep Learning
\item Analyse aktueller Deep Learning-Algorithmen
\end{itemize}

\section{Ziele und Aufgaben}

Folgende zentrale Fragestellungen sollen in der Seminararbeit beantwortet werden:
\begin{itemize}
\item Was ist Deep Learning?
\item Welche Mustertypen können erkannt werden und welche Anwendungen sind möglich
\item Wie sind die gelernten Modelle konkret definiert?
\item Welche Verfahren werden verwendet um vielschichtige Modelle zu lernen?
\item Was ist aktuell mit Deep Learning in Verbindung mit neuronalen Netzen bei der Mustererkennung in Bildern möglich?
\end{itemize}

Ziel dieser Arbeit ist es, dem interessierten Leser einen Einstieg in Themen deep learning, mit besonderem Fokus auf neuronale Netze zu vermitteln. Aufbauend auf diesen Erkenntnissen soll gezeigt werden, welche wesentlichen Algorithmen bis heute entwickelt wurden und welche technischen Problemstellungen mit diesen Algorithmen gelöst werden können oder sogar bereits heute damit gelöst werden. Ein wesentlicher Fokus soll dabei auf die Mustererkennung in Bildern gerichtet werden.

\section{Abgrenzung der Arbeit}
\todo{write}
% TODO
Inhalt...

Achtung, im Rahmen der Verwendung von dinnat und natbib für das Litereaturverzeichnis kann nicht \texttt{$\backslash$cite} verwendet werden, sondern es muss \texttt{$\backslash$citep} verwendet werden!

Beispiel: \citep{Blank2008}, \citep{DeutscheNationalBibliothek2009}, \citep{Ehgartner2004}, \citep{Farkas2007}, \citep{Hatcher2004}, \citep{Henze2006}.
