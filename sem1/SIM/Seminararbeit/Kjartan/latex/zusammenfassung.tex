\chapter{Zusammenfassung}
% 2 Seiten

In der Arbeit wurden die wesentlichen Konzepte und Algorithmen rund um das Thema Deep Learning aufgearbeitet. Die verschiedenen Algorithmen des Deep Learnings basieren auf sehr ähnlichen Prinzipien und wurden mit der Entwicklung der Technik an diese angepasst. So sind moderne Algorithmen, wie der zum Beispiel der DistBelief-Algorithmus, im Wesentlichen darauf ausgelegt eine parallelisierte Berechnung zu ermöglichen und somit die verfügbare Hardware bestmöglich auszunutzen.

Aktuelle Netze sind meist sehr vielschichtig und werden mit unbeaufsichtigtem Lernen vorab trainiert. Häufig erübrigt sich dadurch aufwendiges Feature-Engineering durch Expertenwissen, und wird stattdessen automatisch durch das unbeaufsichtigte Lernen realisiert.

Bereits heute werden aktiv Netze zur Erkennung von Verkehrsschildern, Hausnummern und Gesichtern eingesetzt. Im Grundsatz zeigen momentan Benchmarks, dass die Ergebnisse von neuronalen Netzen im Wesentlichen von der Menge der Trainingsdaten und der Größe eines Netzes abhängen. Durch das Internet und die zunehmende Vernetzung existieren in vielen Bereichen Daten in sehr großen Mengen, die für unüberwachte Lernalgorithmen genutzt werden können. Mit dem Fortschritt der Technik und zunehmendem Interesse an diesem Forschungsgebiet sind bereits in naher Zukunft einige weitere interessante Anwendungen zu erwarten.

Persönlich finde ich die Entwicklungen in Bereich der Chipherstellung besonders interessant. Die Geschichte hat bereits mehrmals gezeigt, dass die Berechnung auf CPUs über Grafikkarten bis hin zu dedizierten Chips enorme Leistungssprünge bringt. So dürfte auch die Entwicklung von dedizierten Chips, wie sie zum Beispiel Qualcomm momentan entwickelt, einige Neuerungen bringen.

Die Leistungsfähigkeit neuronaler Netze ist oft schwierig zu verstehen und so stellt sich mitunter die etwas philosophische Frage, inwieweit ein solches Netz an die Leistungsfähigkeit des Menschen heranreicht oder ihm gar eines Tages Konkurrenz macht. Gut trainiert, schlagen diese Netze den Menschen schon heute in der ein oder anderen Aufgabe. Doch sei an dieser Stelle erwähnt, dass bereits um 1940 die ersten Rechner dieser Welt, die \emph{Zuse} Maschinen von \emph{Konrad Zuse}, den Menschen in vielen Rechenaufgaben übertroffen haben. Bis neuronale Netze so groß werden, dass sie sich ein Bild von der Natur machen können, dürfte noch einige Zeit vergehen.