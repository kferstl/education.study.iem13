\chapter{Zusammenfassung}
% 2 Seiten

In der Arbeit wurden die wesentlichen Konzepte und Algorithmen rund um das Deep Learning aufgearbeitet. Deep Learning-Algorithmen basieren im Wesentlichen auf sehr ähnlichen Prinzipien und wurden mit der Entwicklung der Technik an diese angepasst. So sind moderne Algorithmen, wie der zum Beispiel der DistBelief-Algorithmus, im wesentlichen darauf ausgelegt eine parallelisierte Berechnung zu ermöglichen und somit die Hardware bestmöglich auszunutzen.

Aktuelle Netze sind meist sehr vielschichtig und werden mit unbeaufsichtigtem Lernen vorab trainiert. Aufwendiges Feature-Engineering durch Expertenwissen kann meist schon entfallen und wird automatisch durch das unbeaufsichtigte Lernen gemacht.

Bereits heute werden aktiv Netze zur Erkennung von Verkehrsschildern, Hausnummern und Gesichtern eingesetzt. Im Grundsatz zeigen momentan Benchmarks dass die Ergebnisse von neuronalen Netzen im wesentlichen von der Menge der Trainingsdaten und der Größe des Netzes abhängen. Mit zunehmender Vernetzung und dem Internet existieren in vielen Bereichen Daten in sehr großen Mengen. Mit dem Fortschritt der Technik und zunehmendem Interesse an diesem Forschungsgebiet, sind in naher Zukunft noch viele Interessante Anwendungen zu erwarten.

Persönlich finde ich die Entwicklungen in Bereich der Chipherstellung besonders interessant. Die Geschichte hat bereits mehrmals gezeigt, dass die Berechnung auf CPUs über Grafikkarten bis hin zu dedizierten Chips enorme Leistungssprünge bringt. So dürfte auch die Entwicklung von Chips wie sie zum Beispiel Qualcomm momentan entwickelt einige Neuerungen bringen.

Die Leistungsfähigkeit neuronaler Netze ist schwierig zu verstehen und so entsteht immer wieder die etwas philosophische Frage, in wie weit ein solches Netz an die Leistungsfähigkeit des Menschen kommt oder ihn gar eines Tages Konkurrenz macht. Gut Trainiert schlagen diese Netze Menschen schon heute in der ein oder anderen Aufgabe. An dieser Stelle ist zu erwähnen, dass bereits um 1940 die \emph{Zuse} Maschinen von \emph{Konrad Zuse}, die ersten Rechner der Welt, den Menschen in vielen Rechenaufgaben geschlagen haben. Bis neuronale Netze so groß werden, dass sie sich ein Bild von der Natur machen können, dürfte noch eine ganze Zeit vergehen.